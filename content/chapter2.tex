\section{Discrete Time Signal and System (Time Domain)}
    \subsection{Conventions}
    Considering the continuous signal $x(t)$, we can find it has two elements:
        \begin{itemize}
            \item Independent Variable: It can be time or space, here is time $t$.
            \item Dependent Variable.
        \end{itemize}
    Therefore, we can draw a conclusion: Signal is a function.

    However, a mathmatical function often has analytical expression, for example:
        \begin{equation}
            x(t) = 3t^2
        \end{equation}
    But signals often don't have explicit analytical expressions, 
    so we need to process the signal to get information.

    Now let's introduce continuous-time signal and discrete-time signal:
        \inserttikzpicture
            {
                % --- TikZ 绘图代码开始 ---
                \tikzset{
                    node distance=2.5cm, % 增加了节点间的垂直距离
                    data_node/.style={ % 定义节点的样式
                        font=\normalsize,
                        rectangle,
                        draw,         % 1. 用方框框住
                        align=center, % 1. 文字居中(用于多行)
                        minimum width=7cm, % 确保方框足够宽
                        inner sep=8pt       % 增加一点内边距
                    },
                    arrow_style/.style={ % 定义箭头的样式
                        -Stealth,
                        thick
                    }
                }

                % 1. 定义三个节点,从上到下排列
                % 使用 \\ 来换行
                \node[data_node] (xt) {
                    $x(t)$, $t \in \mathbb{R}$ \\ 
                    using parentheses and variable $t$
                };
                
                \node[data_node, below=of xt] (xnt) {
                    $x(nT)$ \\ 
                    only get values at discrete time point
                };
                \node[data_node, below=of xnt] (xn) {
                    $x[n]$, $n \in \mathbb{Z}$ \\ 
                    using square brakets and variable $n$
                };

                % 2. 绘制箭头并添加标注 (这部分保持不变)
                \draw[arrow_style] (xt) -- node[right, midway] {sampling} (xnt);
                \draw[arrow_style] (xnt) -- node[right, midway] {simplify, variable changes to $n$} (xn);
                % --- TikZ 绘图代码结束 ---
            }
            {Signals}
            {fig:signals}
    
    \newpage
    \subsection{Special Signals}
    Here we introduce some signals.

    \subsubsection{Impulse siganl}
    Definition:
        \begin{equation}
            \delta[n] = 
            \left\{
            \begin{aligned}
                1, \quad & n=0\\
                0, \quad & n\neq 0
            \end{aligned}
            \right.
        \end{equation}
    The most impotant point about this signal is every signal $x[n]$ can be 
    decomposed into the linear combination of shifted impulse $\delta[n-k]$:
        \begin{equation}
            x[n] = \sum_{k=-\infty}^{+\infty} x[k] \underbrace{\delta[n-k]}_{\text{\scriptsize fundamental element}}, k \in \mathbb{Z}
        \end{equation}
    
    And this formula leads to an important idea:
        \inserttikzpicture
            {
                % --- TikZ 绘图代码开始 ---
                \tikzset{
                    node distance=5cm, % 节点之间的水平距离
                    obj_node/.style={ % 定义节点的样式
                        font=\normalsize,
                        rectangle,
                        draw,
                        thick,
                        align=center,
                        minimum width=3.5cm,
                        minimum height=1cm,
                        inner sep=8pt
                    },
                    arrow_style/.style={
                        -Stealth,
                        thick
                    }
                }
                % 1. 定义两个节点,从左到右排列
                \node[obj_node] (complex) {complex object};
                \node[obj_node, right=of complex] (easy) {easy object};
                % 2. 绘制箭头并添加标注
                % 上方箭头 (从左到右)
                % bend left=30 会使箭头向上弯曲
                \draw[arrow_style] (complex) to[bend left=30] 
                    node[above, midway] {decompose} (easy);
                % 下方箭头 (从右到左)
                % bend left=30 会使箭头向下弯曲 (相对于 R->L 的路径)
                \draw[arrow_style] (easy) to[bend left=30] 
                    node[below, midway] {linear combination} (complex);
            }
            {An important Idea}
            {fig:decompose_combine}
    \noindent This idea tells us:
    Comples objects can be decomposed into the linear combination of many easy objects.
    
    \subsubsection{Unit step}
    Definition:
        \begin{equation}
            u[n] = 
            \left\{
            \begin{aligned}
                1, \quad & n \geq 0\\
                0, \quad & n < 0
            \end{aligned}
            \right.
        \end{equation}
    So what is the relation of $u[n]$ and $\delta[n]$? We can easily get:
        \begin{equation}
            \begin{aligned}
                \delta[n] &= u[n] - u[n-1] \\
                u[n]      &= \sum_{k=-\infty}^{n}\delta[k]
            \end{aligned}
        \end{equation}
    
    \subsubsection{an Important Function}
        \begin{equation}
            x_{\omega}(t) = \mathbf{e}^{\mathbf{j}\omega t}
        \end{equation}
    The subscript of $x$ is $\omega$, means the frequency.

    Why do we introduce it? Because if we perform a linear transform on this function, the frequency doesn't change.
    
    And we can give the discrete-time version:
        \begin{equation}
            x_{\omega}[n] = \mathbf{e}^{\mathbf{j}\omega n}
        \end{equation}
    As $n \in \mathbb{Z}$, you can get an interesting property:
        \begin{equation}
            \begin{aligned}
            x_{\omega + 2\pi}[n] &= \mathbf{e}^{\mathbf{j}(\omega+2\pi) n}\\
                                 &= \mathbf{e}^{\mathbf{j}\omega n}
            \end{aligned}
        \end{equation}
    This means, in the frequency domain, this function has periodicity, 
    and the period is $2\pi$.

    What frequency does $2\pi$ here correspond to in the physical world? You will get the answer in the following chapter.

    \subsection{Discrete-Time System}
    We use the following diagram to represent the system and it's function:
        \inserttikzpicture
            {
                % --- TikZ 绘图代码开始 ---
                \tikzset{
                    node distance=2.5cm, % 节点之间的水平距离
                    signal_node/.style={ % 定义信号节点的样式
                        font=\normalsize,
                        minimum width=1cm
                    },
                    system_block/.style={ % 定义系统方框节点的样式
                        font=\normalsize,
                        rectangle,
                        draw,         % 绘制方框
                        thick,        % 方框线条加粗
                        minimum width=2cm,
                        minimum height=1cm,
                        align=center, % 文字居中
                        inner sep=4pt
                    },
                    arrow_style/.style={ % 定义箭头的样式
                        -Stealth,
                        thick,
                        draw % 确保线条被绘制
                    }
                }
                % 1. 定义三个节点,从左到右排列
                \node[signal_node] (input) {$\{x[n]\}_{n=-\infty}^{+\infty}$};
                % 中央的 "system" 节点使用方框样式
                \node[system_block, right=of input] (system) {T};
                
                \node[signal_node, right=of system] (output) {$\{y[n]\}_{n=-\infty}^{+\infty}$};

                % 2. 使用两条直线箭头连接
                % 默认就是直线箭头,无需特殊命令
                \draw[arrow_style] (input) -- (system);
                \draw[arrow_style] (system) -- (output);
                % --- TikZ 绘图代码结束 ---
            }
            {$y[n] = T(\{x[n]\})$}
            {fig:system_diagram}
    Why does we use $\{x[n]\}_{n=-\infty}^{+\infty}$ and $\{y[n]\}_{n=-\infty}^{+\infty}$ to represent the input and output?
    Because the sysetem is divided into:
        \begin{itemize}
            \item Memory: $y[n]$ (output at time point $n$) depends on current and past inputs.
            \item Memoryless: $y[n]$ (output at time point $n$) only depends on current time point input $x[n]$.
        \end{itemize}
    So when we express the function of a system using symbols, a more rigorous way of writiing it is :
        \begin{equation}
            y[n] = T(\{x[n]\})
        \end{equation}
    This means the system $T$ operates on a sequence of inputs, not only one input at time $n$.
    But for simplicity, we write it:
        \begin{equation}
             y[n] = T(x[n])
        \end{equation}
    
    Next, we introduce some typical systems.

    \subsubsection{Delay Device}
        \begin{equation}
            y[n] = x[n-n_d] 
        \end{equation}
    $n_d$ is a scalar.

    How should we understand the function of the system? We can give an example.

    Suppose $n_d$ is 2, and we want to get the output at time 3, that is $y[3]$. 
    According to the formula, we have:
        \begin{equation}
            \begin{aligned}
                y[3] &= x[3-n_d]\\
                     &= x[1]
            \end{aligned}
        \end{equation}
    This means, if we want to get the current ouput, we need to use the past input. 
    So this system is \textbf{memory}. 

    \subsubsection{Integrator}
        \begin{equation}
            y[n] = \sum_{k=-\infty}^{n}x[k]
        \end{equation}
    This formula leads to an important idea:
        \inserttikzpicture
            {
                % --- TikZ 绘图代码开始 ---
                \tikzset{
                    node distance=4cm, % 两个椭圆之间的水平距离
                    op_ellipse/.style={ % 椭圆样式
                        ellipse, 
                        draw, 
                        thick, 
                        minimum width=3cm, 
                        minimum height=4cm,
                        inner sep=10pt % 增加内边距以容纳文字
                    },
                    op_element/.style={ % 椭圆内部元素的样式
                        font=\normalsize, % 使用 \normalsize,如果需要默认正文大小
                        align=center,
                        inner sep=2pt
                    },
                    arrow_style/.style={ % 双向箭头样式
                        <->, % 双向箭头
                        thick
                    }
                }
                % 1. 绘制左侧椭圆 (CT)
                \node[op_ellipse] (ct_ops) {};
                % 在椭圆内部放置元素
                \node[op_element] at (ct_ops.north) [yshift=-1.2cm] (integral) {integral, $\int$};
                \node[op_element, below=0.8cm of integral] (derivative) {derivative, $\frac{d}{dt}$};
                % 2. 绘制右侧椭圆 (DT)
                \node[op_ellipse, right=of ct_ops] (dt_ops) {};
                % 在椭圆内部放置元素,并命名以便连接
                \node[op_element] at (dt_ops.north) [yshift=-1.2cm] (sum) {sum, $\sum$};
                \node[op_element, below=0.8cm of sum] (subtract) {differential, $\nabla$};
                % 3. 连接元素
                \draw[arrow_style] (integral) -- (sum);
                \draw[arrow_style] (derivative) -- (subtract);
                % 4. 在椭圆上方标注文字
                % 为了对齐,我们在椭圆的 north 锚点上方放置标签
                \node[above=0.3cm of ct_ops.north, font=\normalsize] (ct_label) {CT};
                \node[above=0.3cm of dt_ops.north, font=\normalsize] (dt_label) {DT};
                % --- TikZ 绘图代码结束 ---
            }
            {Continuous and Discrete Time Operations}
            {fig:ct_dt_operations}
    \noindent Obviously, Integrator is \textbf{memory}.

    \subsubsection{Square Device}
        \begin{equation}
            y[x] = x^2[n]
        \end{equation}
    Obviously, Integrator is \textbf{memoryless}.

    \subsection{Linearity}
        Linearity is an important property of systems, and also the key to our research.
        
        Let's give the definition from the surface.
        If a system $T$ is a linear system, then:
            \begin{itemize}
                \item Additivity: $T(x[n] + y[n]) = T(x[n]) + T(y[n])$.
                \item Scaling property: $T(\alpha x[n]) = \alpha T(x[n])$.
            \end{itemize}
        (we don't know it's "memory" or "memoryless").

        This definition just gives us the surface input-output relation, but the system is still a black box.
        Now let's analyze the internal structure of the system using \nameref{fig:decompose_combine}.

        Suppose $T$ is a linear system, look at the following operations:
            \inserttikzpicture
                {
                    % --- TikZ 绘图代码开始 ---
                    \tikzset{
                        node distance=3cm, 
                        process_node/.style={ 
                            font=\normalsize,
                            rectangle,
                            draw,
                            thick,
                            minimum width=10cm, 
                            minimum height=1.5cm,
                            align=center,
                            inner sep=8pt
                        },
                        arrow_label/.style={ 
                            font=\normalsize,
                            midway,
                            right
                        },
                        arrow_style/.style={ 
                            -Stealth, % 箭头在路径的 "末端"
                            thick,
                            draw
                        }
                    }

                    % 1. 定义第一个节点
                    \node[process_node] (node1) {$x[n]$};
                    
                    % 3. 定义第二个节点
                    \node[process_node, below=of node1] (node2) {
                        $x[n]=\sum_{k=-\infty}^{+\infty}x[k]\boxed{\delta[n-k]}$
                    };
                    
                    % 5. 定义第三个节点
                    \node[process_node, below=of node2] (node3) {
                        $y[n]=T\left(\sum_{k=-\infty}^{+\infty}x[k]\boxed{\delta[n-k]}\right)$
                    };
                    
                    % 7. 定义第四个节点
                    \node[process_node, below=of node3] (node4) {
                        $y[n] = \sum_{k=-\infty}^{+\infty}x[k]\boxed{T(\delta[n-k])}$
                    };

                    % --- 绘制箭头和标注 ---

                    % 2. 节点1指向节点2的箭头
                    \draw[arrow_style] (node1) -- node[arrow_label] {decompose} (node2);
                    
                    % 6. 节点2指向节点3的箭头 (T is linear)
                    \draw[arrow_style] (node2)-- (node3);
                    
                    % 4. 节点2指向节点3的箭头 (T 作用)
                    \coordinate (midpoint) at ($(node2.south)!.5!(node3.north)$);
                    
                    % --- 这是修正后的代码 ---
                    % 1. 路径从右侧 (midpoint ++(2.5,0)) 指向中点 (midpoint)
                    % 2. arrow_style 的 -Stealth 确保箭头在中点处
                    % 3. node[right, pos=0, ...] 在路径的"起点"(pos=0)的"右侧"放置标签 "T"
                    \draw[arrow_style] (midpoint) ++(2.5,0) 
                        -- node[right, pos=0, font=\normalsize] {T} (midpoint);
                    % --------------------------
                        
                    % 绘制节点3指向节点4的箭头
                    \draw[arrow_style] (node3) --node[arrow_label] {T is linear} (node4);
                    
                    % --- TikZ 绘图代码结束 ---
                }
                {System Process}
                {fig:system_process}

        So we realize the following function:
            \inserttikzpicture
                {
                    % --- TikZ 绘图代码开始 ---
                    \tikzset{
                        % 定义节点间的默认垂直和水平距离
                        node distance=3cm and 6cm, 
                        % 顶部节点的样式 (较窄)
                        top_node/.style={
                            font=\normalsize,
                            rectangle,
                            draw, thick,
                            align=center,
                            minimum width=3cm,
                            minimum height=1.5cm
                        },
                        % 底部节点的样式 (较宽, 以容纳公式)
                        bottom_node/.style={
                            font=\normalsize,
                            rectangle,
                            draw, thick,
                            align=center,
                            minimum width=4.5cm,
                            minimum height=1.5cm
                        },
                        % 箭头样式
                        arrow_style/.style={
                            -Stealth,
                            thick,
                            draw
                        },
                        % 箭头标签样式
                        label_above/.style={font=\normalsize, midway, above},
                        label_right/.style={font=\normalsize, midway, right, align=left}
                    }

                    % --- 1. 定义四个节点 ---
                    
                    % 左上角节点
                    \node[top_node] (TL) {$x[n]$};
                    
                    % 右上角节点
                    \node[top_node, right=of TL] (TR) {$y[n]$};
                    
                    % 左下角节点
                    \node[bottom_node, below=of TL] (BL) {
                        $\sum_{k=-\infty}^{+\infty}x[k]\boxed{\delta[n-k]}$
                    };
                    
                    % 右下角节点 (放置在 TR 的下方,以自动对齐)
                    \node[bottom_node, below=of TR] (BR) {
                        $\sum_{k=-\infty}^{+\infty}x[k]\boxed{T\left(\delta[n-k]\right)}$
                    };

                    % --- 2. 绘制箭头和标注 ---

                    % 2. TL -> TR
                    \draw[arrow_style] (TL) -- 
                        node[label_above] {linear system T} (TR);
                        
                    % 4. TL -> BL
                    \draw[arrow_style] (TL) -- 
                        node[label_right] {decompose\\  using $\delta[n-k]$} (BL);
                        
                    % 6. TR -> BR
                    \draw[arrow_style] (TR) -- 
                        node[label_right] {decompose  \\ using $T(\delta[n-k])$} (BR);
                    
                    % --- TikZ 绘图代码结束 ---
                }
                {Summary of Linear System Derivation}
                {fig:system_derivation_summary}

        That is to say, we decompose the input $x[n]$ into the linear combination of impulses.
        And as the system $T$ is linear, the output can also be decomposed into the linear combinaton, 
        and the elements is impulse response:
                    \begin{equation}
                        h(n,k) = T(\delta[n-k])
                    \end{equation} 
        
        \subsection{Understanding Impulse Response}
        In the above subsection, we get the definition of \textbf{Impulse Response}:
                    \begin{equation}
                        h(n,k) = T(\delta[n-k])
                    \end{equation}
        We give the following obvious understanding:
            \begin{itemize}
                \item $h(n,k)$ is a scalar, when $n$ and $k$ are all determined.
                \item $h(n,k)$ is a function of variable $n$ and $k$.
            \end{itemize}
        Why do we call it impulse response? 
        
        Suppose our input is an impulse at time 2:
            \begin{equation}
                 x[n] = \delta[n-2]
            \end{equation}
        You can get:
            \begin{equation}
                y[n] = T(\delta[n-2]) = h(n,2)
            \end{equation}
        Here, both $\delta[n-1]$ and $h(n,2)$ are signals, not a scalar:
            \inserttikzpicture
                {
                    % --- TikZ/pgfplots 绘图代码开始 ---
                    \pgfplotsset{
                        discrete_plot/.style={
                            ycomb,
                            mark=*,
                            mark size=2pt,
                            line width=1pt,
                            axis lines=middle,
                            axis line style={black, thick},
                            tick label style={font=\small, black},
                            xlabel={$n$}, 
                            xmin=-3.5, xmax=5.5,
                            xtick={-3,-2,-1,0,1,2,3,4,5},
                            enlarge x limits=0.1,
                            enlarge y limits=0.2,
                            title style={yshift=0.5em}, 
                            xlabel style={
                                at={(ticklabel* cs:1.0)},
                                anchor=north west,
                            },
                            ylabel style={
                                at={(ticklabel* cs:1.0)},
                                anchor=west,
                                rotate=0,
                            },
                        }
                    }
                    % ----- 上图: \delta[m-2] -----
                    \begin{axis}[
                        discrete_plot,
                        % --- 2. 修正了这里的花括号 ---
                        ylabel={$x[n]=\delta[n-2]$}, 
                        ymin=-0.2, ymax=1.2, 
                        ytick={1},
                        at={(0, 5.5cm)}
                    ]
                    \addplot+[blue] coordinates {
                        (-3,0) (-2,0) (-1,0) (0,0) (1,0) (2,1) (3,0) (4,0) (5,0)
                    };
                    \end{axis}
                    % ----- 下图: h[m] (随意的火柴杆) -----
                    \begin{axis}[
                        discrete_plot,
                        % --- 3. 修正了这里的花括号 ---
                        ylabel={$y[n]=h(n,2)$}, 
                        ymin=-1.5, ymax=2.5,
                        ytick={-1, 1, 2},
                        at={(0, 0cm)}
                    ]
                    \addplot+[green!60!black] coordinates {
                        (-3,0) (-2,0) (-1,0) (0,1) (1,2) (2,1.5) (3,-1) (4,0) (5,0)
                    };
                    \end{axis}
                    % --- TikZ/pgfplots 绘图代码结束 ---
                }
                {Impulse Response}
                {fig:impulse_response_fixed}
        
        If our input becomes $\delta[n-2] + \delta[n-3]$, what will happen? Our output will become:
                    \begin{equation}
                        y[n] = h(n,2) + h(n,3)
                    \end{equation}
        And if we pay attention to the output at time point 3 or $y[3]$, it's form is:
                    \begin{equation}
                        y[3] = h(3,2) + h(3,3)
                    \end{equation}
        So, in a summary:
                    \begin{itemize}
                        \item $h(n,2)$ means, give an impulse at time 2, the output containing all the time points that we get.
                        \item $h(3,2)$ means, give an impulse at time 2, the output at time 3 that we get.
                    \end{itemize}
        

        \subsection{Linearity-Matrix Perspective}
        Matrix is also a linear transform.

        Suppose we have the following matrix $T$:
                    \begin{equation}
                        \begin{bmatrix}
                            a_{11} & a_{12} & a_{13}\\
                            a_{21} & a_{22} & a_{23}\\
                            a_{31} & a_{32} & a_{33}
                        \end{bmatrix}
                    \end{equation}
        We use this matrix to operate on the following column vector:
             \begin{equation}
                    \begin{bmatrix}
                            x_1\\
                            x_2\\
                            x_3
                    \end{bmatrix}
            \end{equation}
        The results are:
            \begin{equation}
                    % --- 用 \left[ 和 \right] 以及一个 "外部" array 来重建矩阵 ---
                    \left[
                    \begin{array}{c@{\,}c@{\,}c} % c@{\,}c@{\,}c 表示三列,列间加一个"小空格"
                        
                        % 第 1 列
                        \boxed{
                            \begin{array}{@{}c@{}} % "@{}c@{}" 表示一个没有多余边距的内部列
                                a_{11} \\ a_{21} \\ a_{31}
                            \end{array}
                        }
                        & % 外部 array 的下一列
                        
                        % 第 2 列
                        \boxed{
                            \begin{array}{@{}c@{}}
                                a_{12} \\ a_{22} \\ a_{32}
                            \end{array}
                        }
                        & % 外部 array 的下一列

                        % 第 3 列
                        \boxed{
                            \begin{array}{@{}c@{}}
                                a_{13} \\ a_{23} \\ a_{33}
                            \end{array}
                        }
                    
                    \end{array} % 结束外部 array
                    \right]
                    % --- 你的公式的其余部分 ---
                    \begin{bmatrix}
                        x_1\\
                        x_2\\
                        x_3
                    \end{bmatrix}
                    =
                    x_1\begin{bmatrix}
                        a_{11}\\
                        a_{21}\\
                        a_{31}
                    \end{bmatrix} +
                    x_2\begin{bmatrix}
                        a_{12}\\
                        a_{22}\\
                        a_{32}
                    \end{bmatrix}
                    +
                    x_3\begin{bmatrix}
                        a_{13}\\
                        a_{23}\\
                        a_{33}
                    \end{bmatrix}
                \end{equation}
    How can we using the idea above to explain this?

    Actually, we perfrom the following operations:
                \begin{equation}
                    \begin{aligned}
                    T \begin{bmatrix}
                            x_1\\
                            x_2\\
                            x_3
                    \end{bmatrix}
                    &= 
                    T(
                    x_1\begin{bmatrix}
                            1\\
                            0\\
                            0
                    \end{bmatrix}
                    +
                    x_2\begin{bmatrix}
                            0\\
                            1\\
                            0
                    \end{bmatrix}
                    +
                    x_3\begin{bmatrix}
                            0\\
                            0\\
                            1
                    \end{bmatrix}
                    )\\
                    &=
                    x_1T(\begin{bmatrix}
                            1\\
                            0\\
                            0
                    \end{bmatrix})
                    +
                    x_2T(\begin{bmatrix}
                            0\\
                            1\\
                            0
                    \end{bmatrix})
                    +
                    x_3T(\begin{bmatrix}
                            0\\
                            0\\
                            1
                    \end{bmatrix}
                    )             
                    \end{aligned}
                \end{equation}
    We have the similar "impulse":
        \begin{equation}
            \begin{aligned}
                \begin{bmatrix}
                            1\\
                            0\\
                            0
                    \end{bmatrix},\quad
                \begin{bmatrix}
                            0\\
                            1\\
                            0
                    \end{bmatrix},\quad
               \begin{bmatrix}
                            0\\
                            0\\
                            1
                    \end{bmatrix}
            \end{aligned}
        \end{equation}
    And we have the similar "impulse response":
        \begin{equation}
            \begin{aligned}
                T(\begin{bmatrix}
                            1\\
                            0\\
                            0
                    \end{bmatrix}) = \begin{bmatrix}
                        a_{11}\\
                        a_{21}\\
                        a_{31}
                    \end{bmatrix},\quad
                T(\begin{bmatrix}
                            0\\
                            1\\
                            0
                    \end{bmatrix}) =\begin{bmatrix}
                        a_{12}\\
                        a_{22}\\
                        a_{32}
                    \end{bmatrix},\quad
                T(\begin{bmatrix}
                            0\\
                            0\\
                            1
                    \end{bmatrix})=\begin{bmatrix}
                        a_{13}\\
                        a_{23}\\
                        a_{33}
                    \end{bmatrix}
            \end{aligned}
        \end{equation}
    From this example: "impulse response" is a nature of the system itself, has nothing to do with the input.

    \subsection{Time-Invariant}
        Supppose there is an arbitrary system $T$, and we have an input-output combination:
            \inserttikzpicture
            {
                % --- TikZ 绘图代码开始 ---
                \tikzset{
                    node distance=2.5cm, % 节点之间的水平距离
                    signal_node/.style={ % 定义信号节点的样式
                        font=\normalsize,
                        minimum width=1cm
                    },
                    system_block/.style={ % 定义系统方框节点的样式
                        font=\normalsize,
                        rectangle,
                        draw,         % 绘制方框
                        thick,        % 方框线条加粗
                        minimum width=2cm,
                        minimum height=1cm,
                        align=center, % 文字居中
                        inner sep=4pt
                    },
                    arrow_style/.style={ % 定义箭头的样式
                        -Stealth,
                        thick,
                        draw % 确保线条被绘制
                    }
                }
                % 1. 定义三个节点,从左到右排列
                \node[signal_node] (input) {$x[n]$};
                % 中央的 "system" 节点使用方框样式
                \node[system_block, right=of input] (system) {T};
                
                \node[signal_node, right=of system] (output) {$y[n]$};

                % 2. 使用两条直线箭头连接
                % 默认就是直线箭头,无需特殊命令
                \draw[arrow_style] (input) -- (system);
                \draw[arrow_style] (system) -- (output);
                % --- TikZ 绘图代码结束 ---
            }
            {$y[n] = T(\{x[n]\})$}
            {fig:arbitrary_system}
    
    If system $T$ is time-invariant, that is to say:
        \inserttikzpicture
            {
                % --- TikZ 绘图代码开始 ---
                \tikzset{
                    node distance=2.5cm, % 节点之间的水平距离
                    signal_node/.style={ % 定义信号节点的样式
                        font=\normalsize,
                        minimum width=1cm
                    },
                    system_block/.style={ % 定义系统方框节点的样式
                        font=\normalsize,
                        rectangle,
                        draw,         % 绘制方框
                        thick,        % 方框线条加粗
                        minimum width=2cm,
                        minimum height=1cm,
                        align=center, % 文字居中
                        inner sep=4pt
                    },
                    arrow_style/.style={ % 定义箭头的样式
                        -Stealth,
                        thick,
                        draw % 确保线条被绘制
                    }
                }
                % 1. 定义三个节点,从左到右排列
                \node[signal_node] (input) {$x[n-n_d]$};
                % 中央的 "system" 节点使用方框样式
                \node[system_block, right=of input] (system) {T};
                
                \node[signal_node, right=of system] (output) {$y[n-n_d]$};

                % 2. 使用两条直线箭头连接
                % 默认就是直线箭头,无需特殊命令
                \draw[arrow_style] (input) -- (system);
                \draw[arrow_style] (system) -- (output);
                % --- TikZ 绘图代码结束 ---
            }
            {Time-Invariant System $T$}
            {fig:TI_system}
    \noindent Here $n_d$ is a scalar. 
    
    So in a word, if one system $T$ is  time-invariant, 
    then the output delay is the same as the input delay. 

    \subsection{LTI System}
    If a system simultaneously satisfies linearity and time invariance, then we call the system \textbf{LTI System}.

    In the above subsection, if a system $T$ satisfies linearity, 
    then we can give \textbf{Impulse Response}:
        \begin{equation}
            h(n,k) = T(\delta[n-k])
        \end{equation}
    Here $h(n,k)$ is a function of variable $n$ and $k$, and we just know this function has two variables.
    
    Our problem is, if we give the linear system $T$ the time invariance, what form should $h(n,k)$ meet? 
    
    Before the resoning begins, why do we focus on $h(n,k)$? Because here we research the nature of the system,
    and impulse response is the core of this problem.

    Particularly, we give the following definition:
        \begin{equation}
            h[n] = h(n,0) = T(\delta[n])
        \end{equation}
    $h[n]$ means the impulse response of $T$ when we give an impulse at time 0.
    
    This is equivalent to: We give $T$ an input $x[n]=\delta[n]$, the output is $y[n]=h[n]$.

    Now give our input delay of $k$:
        \begin{equation}
            x[n]=\delta[n] \quad \xrightarrow{\mathrm{delay}\quad k} \quad \hat{x}[n]=\delta[n-k]
        \end{equation}

    Suppose system $T$ is time-invariant:
        \begin{equation}
            \hat{y}[n] = y[n-k] = h[n-k]
        \end{equation}

    As:
        \begin{equation}
            h(n,k) = T(\delta[n-k])
        \end{equation}

    So:
        \begin{equation}
            h[n-k] = h(n,k)
        \end{equation}
    
    \subsection{Convolution}
    Now we know, if a system $T$ satisfies linearity and time-invariance, give it an input $x[n]$, 
    output is:
        \begin{equation}
            y[n] = \sum_{k=-\infty}^{+\infty} x[k]h[n-k]
        \end{equation}
    
    We can give the definition of two signals' $\ast$ oepration:
        \begin{equation}
            x[n] \ast h[n] = \sum_{k=-\infty}^{+\infty} x[k]h[n-k]
        \end{equation}
    And we can say: If a system is LTI system, then:
        \begin{equation}
            \mathrm{output} = \mathrm{input} \ast  \mathrm{impulse \quad response}
        \end{equation}
    
    \subsection{An Example of Convolution}
    Suppose $T$ is a LTI system, and the impulse response is:
        \begin{equation}
            h[n] = u[n] - u[n-N]
        \end{equation}
    here $u[n]$ is unit step, $N$ is a positive integer.
    We give it an input:
        \begin{equation}
            x[n] = a^nu[n]
        \end{equation}
    here $|a|<1$.

    Calsulate the output:
        \begin{equation}
            y[n] = x[n] \ast h[n] = \sum_{k=-\infty}^{+\infty} x[k]h[n-k]
        \end{equation}
    
    \subsubsection{Method1. Analytical Calculations.}

    Expand the convolution formula:
        \begin{equation}
            \begin{aligned}
            y[n] &= \sum_{k=-\infty}^{+\infty} x[k]h[n-k]\\
                 &= \sum_{k=-\infty}^{+\infty} a^ku[k]u[n-k] - \sum_{k=-\infty}^{+\infty}a^ku[k]u[n-N-k]
            \end{aligned}
        \end{equation}
    
    As the unit step $u[n]$ gives 0 when $n<0$, and $k$ is summation variable, we need to detemine which part of $k$ gives a non-zero result.

    For the part $\sum_{k=-\infty}^{+\infty} a^ku[k]u[n-k]$, if it's non-zero, then:
        \begin{equation} \label{k_range_1}
            k \in [0,n]
        \end{equation}

    For the part $\sum_{k=-\infty}^{+\infty} a^ku[k]u[n-N-k]$, if it's non-zero, then:
        \begin{equation}\label{k_range_2}
            k \in [0,n-N]
        \end{equation}
    
    Then because our main variable is $n$, we need to have a segmented discussion about different range of $n$, 
    and use the \eqref{k_range_1} and \eqref{k_range_2} to find whether the part is 0 or not.

    For $n \in (-\infty,0)$, the range of $k$ cann't meet \eqref{k_range_1} or \eqref{k_range_2}, so both part are 0:
        \begin{equation}
            y[n] = 0
        \end{equation}
    
    For $n \in [0,N-1]$, the range of $k$ can meet \eqref{k_range_1}, but cann't meet \eqref{k_range_2}:
        \begin{equation}
            \begin{aligned}
            y[n] &= \sum_{k=-\infty}^{+\infty} a^ku[k]u[n-k]\\
                 &= \sum_{k=0}^{n}a^k
            \end{aligned}
        \end{equation}
    
    For $n \in [N,\infty]$, the range of $k$ can meet \eqref{k_range_1}, and can meet \eqref{k_range_2}:
        \begin{equation}
            \begin{aligned}
            y[n] &= \sum_{k=-\infty}^{+\infty} a^ku[k]u[n-k] -\sum_{k=-\infty}^{+\infty} a^ku[k]u[n-N-k] \\
                 &= \sum_{k=0}^{n}a^k - \sum_{k=0}^{n-N}a^k
            \end{aligned}
        \end{equation}
    
    \newpage
    \subsubsection{Graphical Solution}
    
    The graph of $h[n]$ and $x[n]$ are as follows, here we choose $a = 1/2$:
        \inserttikzpicture
            {
                % --- TikZ/pgfplots 绘图代码开始 ---
                \begin{groupplot}[
                    % --- 1. 使用 groupplot 来自动对齐和布局 ---
                    group style={
                        group size=1 by 2, % 1 列, 2 行
                        vertical sep=0.0cm  % 自动处理间距,保证不重叠
                    },
                    % --- 这是“轴 (Axis)”的样式 (保持不变) ---
                    axis lines=middle,
                    axis line style={black, thick},
                    tick label style={font=\small, black},
                    xlabel={$n$},
                    enlarge y limits=0.2,
                    xmin=-7.5, xmax=7.5,
                    xlabel style={
                        at={(ticklabel* cs:1.0)},
                        anchor=north west,
                    },
                    ylabel style={
                        at={(ticklabel* cs:1.0)},
                        anchor=west,
                        rotate=0,
                    },
                    % --- 这是修正的地方 ---
                    % 将“绘图 (Plot)”的样式移动到这里
                    every axis plot post/.style={
                        ycomb,
                        mark=*,
                        mark size=2pt,
                        line width=1pt,
                    }
                    % --- 修正结束 ---
                ]
                % ----- 1. 上图: u[n] - u[n-N] -----
                \nextgroupplot[
                    ylabel={$h[n]=u[n] - u[n-N]$},
                    ymin=-0.2, ymax=1.2, 
                    ytick={1},
                    xtick={6},
                    xticklabels={$(N-1)$}
                ]
                % \addplot+ 会自动继承 "every axis plot post" 样式
                \addplot+[blue] expression[domain=-2:6, samples=9] {
                    (x>=0 && x<7) ? 1 : 0
                };
                % ----- 2. 下图: a^n u[n] -----
                \nextgroupplot[
                    ylabel={$x[n]=a^n u[n], (a=1/2)$},
                    ymin=-0.2, ymax=1.2,
                    ytick={0.5, 1},
                    xtick={6},
                    xticklabels={$(N-1)$}
                ]
                % \addplot+ 会自动继承 "every axis plot post" 样式
                \addplot+[green!60!black] expression[domain=-6:6, samples=13] {
                    (x>=0) * ( (1/2)^x )
                };
                \end{groupplot}
                % --- TikZ/pgfplots 绘图代码结束 ---
            }
            {$h[n]$ and $x[n]$}
            {fig:x_n_h_n}    

    As the summation variable is $k$, we perform the following operations:
                \begin{equation}
                    h[n-k] = h[-(k-n)]
                \end{equation}
    How can we get $h[-(k-n)]$? We can use $h[-k]$ and shift $n$ units:
        \inserttikzpicture
            {
                % --- TikZ/pgfplots 绘图代码开始 ---
                \begin{groupplot}[
                    % --- 1. groupplot 样式 ---
                    group style={
                        group size=1 by 2,
                        vertical sep=0.0cm 
                    },
                    % --- 2. 统一的轴 (Axis) 样式 ---
                    axis lines=middle,
                    axis line style={black, thick},
                    tick label style={font=\small, black},
                    xlabel={$k$},
                    enlarge y limits=0.2,
                    % --- 3. 关键:新的统一 X 轴范围 ---
                    xmin=-7.5, xmax=7.5,
                    % --- 统一的 xlabel/ylabel 样式 ---
                    xlabel style={
                        at={(ticklabel* cs:1.0)},
                        anchor=north west,
                    },
                    ylabel style={
                        at={(ticklabel* cs:1.0)},
                        anchor=west,
                        rotate=0,
                    },
                    % --- 4. 统一的绘图 (Plot) 样式 ---
                    every axis plot post/.style={
                        ycomb,
                        mark=*,
                        mark size=2pt,
                        line width=1pt,
                    }
                ]
                % ----- 1. 上图: h[n] = u[n] - u[n-N] -----
                \nextgroupplot[
                    ylabel={$h[k]$},
                    ymin=-0.2, ymax=1.2, 
                    ytick={1},
                    xtick={6}, % 只在 n=6 处显示刻度
                    xticklabels={$(N-1)$} % 并将其标记为 N-1
                ]
                % (x>=0 && x<7) 模拟 u[n] - u[n-7] (即 N-1=6)
                \addplot+[blue] expression[domain=-6:6, samples=13] {
                    (x>=0 && x<7) ? 1 : 0
                };
                % ----- 2. 下图: h[-n] (关于纵轴对称) -----
                \nextgroupplot[
                    ylabel={$h[-k]$}, % 绘制 h[-n]
                    ymin=-0.2, ymax=1.2,
                    ytick={1},
                    xtick={-6}, % 只在 n=-6 处显示刻度
                    xticklabels={$-(N-1)$} % 并将其标记为 -(N-1)
                ]
                % (x<=0 && x>-7) 是 h[-n] 的逻辑
                \addplot+[green!60!black] expression[domain=-6:6, samples=13] {
                    (x<=0 && x>-7) ? 1 : 0
                };
                
                \end{groupplot}
                % --- TikZ/pgfplots 绘图代码结束 ---
            }
            {Top: The signal $h[n]$ and time-reversed signal $h[-n]$.}
            {fig:signals_reversed}
        
        Then shift $h[-k]$ by $n$ units, and perform corresponding multiplication:
                \inserttikzpicture
                    {
                        % --- TikZ/pgfplots 绘图代码开始 ---
                        \begin{groupplot}[
                            % --- 1. groupplot 样式 ---
                            group style={
                                group size=1 by 2,
                                vertical sep=0.0cm  % 增加了垂直间距
                            },
                            % --- 2. 统一的轴 (Axis) 样式 ---
                            axis lines=middle,
                            axis line style={black, thick},
                            tick label style={font=\small, black},
                            xlabel={$k$}, % <-- 关键:横轴是 k
                            enlarge y limits=0.2,
                            % --- 3. 统一 X 轴范围 (来自上一轮) ---
                            xmin=-7.5, xmax=7.5,
                            % --- 统一的 xlabel/ylabel 样式 ---
                            xlabel style={
                                at={(ticklabel* cs:1.0)},
                                anchor=north west,
                            },
                            ylabel style={
                                at={(ticklabel* cs:1.0)},
                                anchor=west,
                                rotate=0,
                            },
                            % --- 4. 统一的绘图 (Plot) 样式 ---
                            every axis plot post/.style={
                                ycomb,
                                mark=*,
                                mark size=2pt,
                                line width=1pt,
                            }
                        ]
                        % ----- 1. 上图: h[n-k] for n=2 -----
                        \nextgroupplot[
                            ylabel={$h[-(k-2)]$}, % h[n-k] 当 n=2
                            ymin=-0.2, ymax=1.2, 
                            ytick={1},
                            xtick={-6, 0, 6}, % 标注关键点
                            xticklabels={$-(N-1)$, $0$, $(N-1)$} % N-1=6
                        ]
                        % (x>=-4 && x<=2) 是 h[2-k] 的逻辑
                        \addplot+[blue] expression[domain=-6:6, samples=13] {
                            (x>=-4 && x<=2) ? 1 : 0
                        };
                        % ----- 2. 下图: x[k] = a^k u[k] -----
                        \nextgroupplot[
                            ylabel={$x[k] = a^k u[k]$},
                            ymin=-0.2, ymax=1.2,
                            ytick={0.5, 1},
                            xtick={-6, 0, 6}, % 保持刻度与上图对齐
                            xticklabels={$-(N-1)$, $0$, $(N-1)$}
                        ]
                        % (x>=0) * ... 模拟了 u[k]
                        \addplot+[green!60!black] expression[domain=-6:6, samples=13] {
                            (x>=0) * ( (1/2)^x )
                        };
                        
                        \end{groupplot}
                        % --- TikZ/pgfplots 绘图代码结束 ---
                    }
                    { $y[n] = \sum x[k]h[n-k]$ at $n=2$. }
                    {fig:convolution_step}
    
                
    
    
    
    
    
    

    
    
    
    











