\section{Introduction}
    \subsection{Some Suggestions}
        In the first subsection, let's give the following suggestions.

        \noindent \textcolor{red}{No reading, No learning.}

        After we enter a high level of learning, we mainly depend on reading,
        because no one will tell us the key points. 
        The reason for setting up a series of courses is to make our learning more targeted.

        \noindent \textcolor{red}{No writing, No reading.}

        What is effective reading? You must write something in the paper, such as the math notations.
        Only in that way will you brain not skip many details.
        
        \noindent \textcolor{red}{No data, No truth.}

        In the vast majority of scientific fields, 
        we must apply theroies and methods to real data rather than relying on simulation.
        
        \noindent \textcolor{red}{No Analytic, No understanding.}

        If you say, "You have well understood something", you must provide a detailed analysis
        {using symbolic language}.
        
        \noindent \textcolor{red}{No programming, No coginition.}

        Finally, if you wish to have an intutive coginition of something, 
        the typical approach is to use a series of visualization methods (such as pictures, tables...) to present the images in your mind.

    \subsection{Digital vs Analog}
        There are the following differences between the two.

        First, digital means \textcolor{red}{programmable}. 
        Analog rely on hardware, digital are true software, which is very flexible and easy to change.

        Second, in the digital world we use bit to describe precision. 
        And digital has a high and controllable precision.
        
        Third, storing, digital has a high storage density and compressible.

        Finally, digital's cost is low, because it's manufactured on a silicon wafer.

    \newpage
    \subsection{History of DSP}
        Now let's briefly review the history of DSP.

        \inserttikzpicture
            {
                %  1. 参数定义 
                \def\amplitude{1.0}      % 振幅 (cm)
                \def\quarterperiod{2.5}  % 1/4 周期在y轴上的长度 (cm)
                \def\numsteps{6}         % 节点总数 - 1 
                \def\linelength{1.5}     % 节点到文字框的短横线长度 (cm)
                \def\eventwidth{6.5cm}   % 事件矩形框的宽度
                            
                % 2. 节点样式定义 
                % 事件矩形框
                \tikzstyle{eventbox} = [
                    rectangle, 
                    draw, 
                    semithick, 
                    text width=\eventwidth, 
                    align=left,             
                    inner sep=5pt
                    ]

                %  3. 数据 (共7个)
                \def\eventlist{{"\textbf{\textcolor{red}{DSP (processor).}}\\
                                A special CPU \\ designed for signal processing task.", 
                                "\textbf{\textcolor{red}{Cooley and Tukey, FFT.}}\\
                                The origin of signal processing.", 
                                "\textbf{\textcolor{red}{The Apollo Moon-Landing Project.}}\\
                                Giving rise to Kalman Filtering, \\
                                which is another methods of Optimal Filtering.", 
                                "\textbf{\textcolor{red}{Word War 2.}}\\
                                Wiener Filtering (the foundation of Optimal Filtering)\\
                                Nyquist-Shannon Sampling Theorem", 
                                "\textbf{\textcolor{red}{Bode, Control Science.}}\\
                                In the field of control science, the first step is perception. \\
                                Perception is sampling to get data, and extract useful information from the data.", 
                                "\textbf{\textcolor{red}{Marconi, Electromagnetic waves are simple harmonics.}}\\
                                This discovery extends Fourier Transform form thermal to electromagnetic.", 
                                "\textbf{\textcolor{red}{Fourier, Fourier Transform.}}\\
                                any complex motion $\longleftrightarrow$ linear combination of simple harmonic motion."}}
                            
                \def\timelist{{ "1980",
                                "1960", 
                                "1950", 
                                "1940", 
                                "1930", 
                                "1900", 
                                "1730"}}
                                
                % 4. 计算 
                \pgfmathsetmacro{\period}{\quarterperiod * 4}
                \pgfmathsetmacro{\degfactor}{360 / \period}
                \pgfmathsetmacro{\totalheight}{\numsteps * \quarterperiod}

                % 5. 绘制基础时间轴 
                \draw [dotted, semithick, line cap=round]
                plot [domain=0:\totalheight, samples=200, variable=\y]
                ({\amplitude * sin(\degfactor * \y)}, \y);
                            
                \foreach \i in {0, 1, ..., \numsteps}
                {
                    \pgfmathsetmacro{\ypos}{\i * \quarterperiod}
                    \pgfmathsetmacro{\xpos}{\amplitude * sin(\degfactor * \ypos)}
                    \fill (\xpos, \ypos) circle (2pt);
                }

                % 6. 绘制交错的标签     
                \foreach \i in {0, 1, ..., \numsteps}
                {
                    \pgfmathsetmacro{\ypos}{\i * \quarterperiod}
                    \pgfmathsetmacro{\xpos}{\amplitude * sin(\degfactor * \ypos)}
                    \pgfmathsetmacro{\eventtext}{\eventlist[\i]}
                    \pgfmathsetmacro{\timetext}{\timelist[\i]}
                    \pgfmathparse{mod(\i,2)}
                    \ifdim\pgfmathresult pt=0pt % 偶数 (0, 2, 4, 6), 放在右侧
                            
                    % 在绘制短线的同时, 使用 'node' 在其上方 (above) 中点 (pos=0.5) 放置时间
                    \draw [semithick] (\xpos, \ypos) -- 
                    node [above, pos=0.5, yshift=1pt] {\timetext} 
                    ++(\linelength, 0);
                                
                    % 放置事件矩形 (不变)
                    \node (event) [eventbox, anchor=west] 
                    at (\xpos + \linelength, \ypos) {\eventtext};

                    \else % 奇数 (1, 3, 5), 放在左侧

                    \draw [semithick] (\xpos, \ypos) -- 
                    node [above, pos=0.5, yshift=1pt] {\timetext} 
                    ++(-\linelength, 0);
                                
                    % 放置事件矩形 (不变)
                    \node (event) [eventbox, anchor=east] 
                    at (\xpos - \linelength, \ypos) {\eventtext};
                                
                    \fi
                }
            }
            {History of DSP}
            {fig:history_dsp}


        


        



        



        