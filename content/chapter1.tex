\setchapterquote{Talk is cheap. Show me the code.}{Linus Torvalds}
\section{Introduction}
    \subsection{数字 vs 模拟}
    在数字世界中,我们使用比特(bit)来描述精度。
    这赋予了它高且可控的精度。

    数字意味着\textcolor{red}{可编程}。
    不同于模拟依赖于硬件,数字非常灵活且易于更改。

    数字是制造在硅片上的,具有高存储密度且可压缩。
    并且成本很低。


    \subsection{DSP的历史}
    现在让我们简要回顾一下DSP的历史。

    \inserttikzpicture
        {
            \begin{scope}[scale=0.65]
            % --- 1. 参数调整 (关键修改) ---
            % 将宽度改小以适应页面
            \def\amplitude{0.7}      % 振幅: 1.0 -> 0.7 (让波浪更瘦)
            \def\quarterperiod{2.5}  % 纵向间距保持不变
            \def\numsteps{6}         % 节点数
            \def\linelength{1.5}     % 连接线: 1.5 -> 0.6 (拉近文字与轴的距离)
            \def\eventwidth{5.5cm}   % 文本宽: 6.5cm -> 4.8cm (核心:防止溢出)
                        
            % 2. 节点样式定义 
            \tikzstyle{eventbox} = [
                rectangle, 
                draw, 
                semithick, 
                text width=\eventwidth, 
                align=left,            
                inner sep=5pt,
                font=\footnotesize % 稍微调小字号以适应窄框
                ]

            % 3. 数据 (保持不变)
            \def\eventlist{{
                "\textbf{\textcolor{red}{DSP (processor).}}\\
                A special CPU \\ designed for signal processing task.", 
                "\textbf{\textcolor{red}{Cooley and Tukey, FFT.}}\\
                The origin of signal processing.", 
                "\textbf{\textcolor{red}{The Apollo Moon-Landing Project.}}\\
                Giving rise to Kalman Filtering, \\
                which is another methods of Optimal Filtering.", 
                "\textbf{\textcolor{red}{Word War 2.}}\\
                Wiener Filtering (the foundation of Optimal Filtering)\\
                Nyquist-Shannon Sampling Theorem", 
                "\textbf{\textcolor{red}{Bode, Control Science.}}\\
                In the field of control science, the first step is perception. \\
                Perception is sampling to get data, and extract useful information from the data.", 
                "\textbf{\textcolor{red}{Marconi, Electromagnetic waves are simple harmonics.}}\\
                This discovery extends Fourier Transform form thermal to electromagnetic.", 
                "\textbf{\textcolor{red}{Fourier, Fourier Transform.}}\\
                $x(t) \longleftrightarrow \sum_{n=-\infty}^{+\infty}a_n\mathrm{e}^{\mathrm{j}n\omega_0 t}$\\
                Any complex motion $\longleftrightarrow$ linear combination of simple harmonic motion."
            }}
                        
            \def\timelist{{ 
                "1980", "1960", "1950", "1940", 
                "1930", "1900", "1730"
            }}
                            
            % 4. 计算 
            \pgfmathsetmacro{\period}{\quarterperiod * 4}
            \pgfmathsetmacro{\degfactor}{360 / \period}
            \pgfmathsetmacro{\totalheight}{\numsteps * \quarterperiod}

            % 5. 绘制基础时间轴 
            \draw [dotted, semithick, line cap=round]
            plot [domain=0:\totalheight, samples=200, variable=\y]
            ({\amplitude * sin(\degfactor * \y)}, \y);
                        
            \foreach \i in {0, 1, ..., \numsteps}
            {
                \pgfmathsetmacro{\ypos}{\i * \quarterperiod}
                \pgfmathsetmacro{\xpos}{\amplitude * sin(\degfactor * \ypos)}
                \fill (\xpos, \ypos) circle (2pt);
            }

            % 6. 绘制交错的标签     
            \foreach \i in {0, 1, ..., \numsteps}
            {
                \pgfmathsetmacro{\ypos}{\i * \quarterperiod}
                \pgfmathsetmacro{\xpos}{\amplitude * sin(\degfactor * \ypos)}
                \pgfmathsetmacro{\eventtext}{\eventlist[\i]}
                \pgfmathsetmacro{\timetext}{\timelist[\i]}
                \pgfmathparse{mod(\i,2)}
                
                % 修正:确保 if 判断比较的是整数
                \ifnum\pdfstrcmp{\pgfmathresult}{0.0}=0 % 偶数 (右侧)
                        
                \draw [semithick] (\xpos, \ypos) -- 
                node [above, pos=0.5, yshift=1pt] {\timetext} 
                ++(\linelength, 0);
                            
                \node (event) [eventbox, anchor=west] 
                at (\xpos + \linelength, \ypos) {\eventtext};

                \else % 奇数 (左侧)

                \draw [semithick] (\xpos, \ypos) -- 
                node [above, pos=0.5, yshift=1pt] {\timetext} 
                ++(-\linelength, 0);
                            
                \node (event) [eventbox, anchor=east] 
                at (\xpos - \linelength, \ypos) {\eventtext};
                            
                \fi
            }
            \end{scope}
        }
        {History of DSP}
        {fig:history_dsp}