\section{Introduction}
    \subsection{Overview}
        Some prerequisite:


        In order to study this course better, 
        you need to have some mathmatical maturity and a foundation of calculus.
        
        The textbook we use is \textcolor{red}{Discrete-Time Signal Processing, 3rd edition} written by A. V. Oppenheim.


    \subsection{Some Suggestions}
        In the first subsection, let's give the following suggestions.

        \noindent \textcolor{red}{No reading, No learning.}

        After we enter a high level of learning, we mainly depend on reading,
        because no one will tell us the key points. 
        The reason for setting up a series of courses is to make our learning more targeted.

        \noindent \textcolor{red}{No writing, No reading.}

        What is effective reading? You must write something in the paper, such as the math notations.
        Only in that way will you brain not skip many details.
        
        \noindent \textcolor{red}{No data, No truth.}

        In the vast majority of scientific fields, 
        we must apply theroies and methods to real data rather than relying on simulation.
        
        \noindent \textcolor{red}{No Analytic, No understanding.}

        If you say, "You have well understood something", you must provide a detailed analysis
        {using symbolic language}.
        
        \noindent \textcolor{red}{No programming, No coginition.}

        Finally, if you wish to have an intutive coginition of something, 
        the typical approach is to use a series of visualization methods (such as pictures, tables...) to present the images in your mind.

    \subsection{Digital vs Analog}
        There are the following differences between the two.

        First, digital means \textcolor{red}{programmable}. 
        Analog rely on hardware, digital are true software, which is very flexible and easy to change.

        Second, in the digital world we use bit to describe precision. 
        And digital has a high and controllable precision.
        
        Third, storing, digital has a high storage density and compressible.

        Finally, digital's cost is low, because it's manufactured on a silicon wafer.

    \subsection{Course Arrangement}
        The digital signal processing course has the following chapters.

        \noindent \textcolor{red}{Chapter1: Preliminary for Digital.} 

        In this chapter, we will introduce {Discrete Signal and System} from the following two perspectives:
            \begin{itemize}
                \item Time domain;
                \item Frequency domain.
            \end{itemize}
        
        \noindent \textcolor{red}{Chapter2: How to obtain digital signal.} 
        
        In this chapter, we will introduce:
            \begin{itemize}
                \item A/D: Using sampling to help us move from the real word (continuous) to digital word.
                \item D/A: Return journey.
            \end{itemize}
        
        
        \noindent \textcolor{red}{Chapter3: How to process digital signal.} 

        In this chapter, we have two basic tools for recoginzing signals:
            \begin{itemize}
                \item Fourier Transform.
                \item Z transform.
            \end{itemize}
        Also we will introduce Linear Filtering;
            \begin{itemize}
                \item Representation: Discrete Convolution.
                \item Implemention: Time domain $\longleftrightarrow$ Frequency domain.
                \item Architecture: from software $ \longrightarrow$ chip.
            \end{itemize}
        
        \newpage
        \noindent \textcolor{red}{Chapter4: How to improve performance.}

        We have the following pursuit:
            \begin{itemize}
                \item Faster: FFT.
                \item More accurate
                    \begin{itemize}
                        \item Quantization Noise.
                        \item Finite word length.
                    \end{itemize}
            \end{itemize}
        
        \noindent \textcolor{red}{Chapter5: How to design linear filters.}

        We will introduce the following classic filters:
            \begin{itemize}
                \item FIR.
                \item IIR.
                \item Hilbert transform.
            \end{itemize}
        
        \noindent \textcolor{red}{Chapter6: How to extend linear filters.}

        We will introduce:
            \begin{itemize}
                \item Real signal $\longrightarrow$ Complex signal.
                \item Multirate filters.
            \end{itemize}
        
        \noindent \textcolor{red}{Chapter7: How to apply linear filters.}
    
    \newpage
    \subsection{History of DSP}
        Now let's briefly review the history of DSP.

        \noindent \textcolor{red}{1730, Fourier, Fourier Transform.}

        In the process of studying the heat conduction, 
        Fourier discovered that:
            \begin{center}
                any complex motion $\longleftrightarrow$ linear combination of simple harmonic motion
            \end{center}
        and that is the Fourier Transform.

        \noindent \textcolor{red}{1900, Marconi, Electromagnetic waves are simple harmonics}

        This discovery extends Fourier Transform form thermal to electromagnetic.

        \noindent \textcolor{red}{1930, Bode, Control Science.}

        In the field of control science, the first step is perception. 
        Perception is sampling to get data, and extract useful information from the data.
        
        \noindent \textcolor{red}{1940, Word War 2.}

        During this period, two important discoveries were made:
            \begin{itemize}
                \item Wiener Filtering: the foundation of Optimal Filtering (Stochastic Process required).
                \item Nyquist-Shannon Sampling Theorem.
            \end{itemize}
        
        \noindent \textcolor{red}{1950, The Apollo Moon-Landing Project.}

        This plan gave rise to Kalman Filtering, another methods of Optimal Filtering.

        \noindent \textcolor{red}{1960, Cooley and Tukey, FFT.}

        This is the origin of signal processing.

        \noindent \textcolor{red}{1980, DSP (processor).}

        A special CPU designed for signal processing task.


        


        



        



        